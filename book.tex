% Documentation for HA-STS, my home automation project.
%
% To build this document, run the following:
% latex book && latex book && dvipdfm book
\documentclass[letterpaper,twoside,12pt]{book}
\usepackage{fancyhdr}
\usepackage[letterpaper]{geometry}
%\usepackage{lastpage}
%\usepackage{tabularx}
%\usepackage{multirow}
%\usepackage{graphicx}
%\usepackage{color}
%\usepackage{float}
%\usepackage[section]{placeins}
\usepackage[section=subsection,numberedsection=nolabel,nonumberlist]{glossaries}

% Document metadata
\title{HA-STS \\ The automation of my home.}
\author{Daniel Williams \\ \texttt{dwilliams@port8080.net}}
\date{November 2013 - Present}

% Format the headers and footers

% Make the glossaries and add entries.  This should probably be broken out into
% another file at some point.
\makeglossaries
%\renewcommand*{\glossaryname}{Definitions and Acronyms}

\newacronym{aaa}{AAA}{Authentication, Authorization, and Accounting}
\newglossaryentry{openbsd}{
    name={OpenBSD},
    description={An operating system considered by many to be the most secure
    mainstream operating system pulicly available.  For more information, please
    see the website at http://www.openbsd.org.}
}
\newglossaryentry{openntpd}{
    name={OpenNTPD},
    description={A time server implementation using \gls{NTP} as its transport
    protocol.  This implementation is a free and simplified alternate to the
    reference implementation which requires fewer resources to run.  For more
    information, please see the website at http://www.openntpd.org.}
}
\newacronym{ntp}{NTP}{Network Time Protocol}

\begin{document}
\frontmatter
\maketitle

\chapter{Preface}
This book is being written to plan, guide, and document the trials and
tribulations of designing and building home automation systems.  Throughout
this process, best practices and open standards will be followed as much as
possible.  Where there are no best practices or open standards, suggestions will
be documented in hopes of guiding others.

\mainmatter
\tableofcontents

\chapter{Introduction}
% Will put as much as possible on OpenBSD for security.

%\section{Who am I?}

\section{What is to be accomplished here?}
% Building the systems component by component
% Small, inexpensive, non-invasive
% If possible, energy efficent
%   this will focus primarily on lighting

\section{Constraints and Considerations}
% SECURITY SECURITY SECURITY
% This is a home, make it as SECURE as possible
% OpenBSD if possible
% Review the man pages as they are THE documentation
% COTS to keep price down

\section{License and Disclaimers}
% Because it should be done
% Simplified BSD
% Any other "you're a dumbass" disclaimers

% Chapter - Authentication, Authorization, and Accounting
\chapter{Authentication, Authorization, and Accounting}
% Explain AAA and why we need it

\section{Dependencies}

\subsection{OpenNTPD}
% Talk about a basic OpenNTPD setup
One of the key components to any \gls{aaa} system is time.  Determining when a
users' access is valid is integral to maintaining a secure system.  This helps
to prevent replay attacks and prevent access using past (or future) expired
credentials.

\gls{openbsd} prefers the \gls{openntpd} implementation of the \gls{ntp} service
because it is a free (as in freedom) and simple implementation.  Initially,
\gls{openntpd} will be setup on the \gls{aaa} server in order to keep the
\gls{aaa} services better in sync with the clock.  In the future, more advanced
time servers will be setup.  At that point, the \gls{openntpd} server on the
\gls{aaa} server will be setup to sync from those time servers.

In order to activate a basic setup of \gls{openntpd} on \gls{openbsd}, the
following line should be entered into the \verb=/etc/rc.conf.local= file:
\begin{verbatim}
ntpd_flags=
\end{verbatim}
Upon reboot, the \gls{openntpd} service will start automatically with the system
and keep the clock in sync with the internet time servers.

% Do I add a subsection here or later on for more advanced OpenNTPD topics?
% Things like alternate time sources: GPS, Atomic Radio, Local Atomic (rubidium)

\subsection{Domain Name Service}
% We'll setup a DNS server on the kerberos domain controller.
Another key component to any \gls{aaa} system is name resolution.  Name
resolution allows client machines to find the key server to request credentials
for access.  Without being able to find the key server, the key server is rather
useless.  A number of people will say that it's better to hard code the IP
address of the key server into the clients, but DNS keeps us from having to
manually update that configuration.  Hard set IP addresses are not more secure
than DNS, as both can be spoofed and modified on the network.

% At some future point, move to another server with CARP and management scripts.

\subsection{Dynamic Host Configuration Protocol}
% Basic DHCP setup.  Should also make it handle IPv6 and putting the names into
% DNS.

% At some future point, move to another server with CARP and management scripts
% (probably the same servers as DNS)

\section{Kerberos 5}
% Mention that Kerberos serves the Authentication role
% Discuss a basic kerberos setup

% Do I add subsections here for things like redundancy (CARP) here?

\section{Network Information Service}
% Mention that NIS/YP server the Authorization role
% Lets use NIS/YP over LDAP

% Do I add subsections here for things like redundancy (CARP) here?

\section{Accounting - ??}
% Is syslog sufficient?

% Do I add subsections here for things like redundancy (CARP) here?

\section{Certificate Authority}
% Does it make sense to put this here?  Most of our connections will be secured
% via SSL, so we should have a CA



% Chapter - Configuration Management
\chapter{Configuration Management}

\section{Dependencies}

\subsection{Source Code Management}
% Not really a configuration management item, but it's a dependency
% I'm liking git-o-lite for non-public repositories.

\subsection{Artifact Management}
% Not really a configuration management item, but it's a dependency

\section{Puppet Master}
% Puppet master on OpenBSD

\section{Pupper Slaves}

\subsection{OpenBSD}
% Other BSDs?

\subsection{Linux}
% Debian or CentOS?



% Chapter - Data Handling and Storage
\chapter{Data Handling and Storage}

\section{Unified Home Spaces}
% Export /home over the network (NFS, CIFS, SFTP?, HTTP?)

\section{Media Serving}
% Read-only big storage (CIFS, PMS, HTTP?)

% Might not be a bad idea to mention things like Plex, Drobo, and FreeNAS.

\section{Log Archival}
% High compression with log searching, trending



% Chapter - Communication
\chapter{Communication}

\section{Firewall}
% Setup an OpenBSD PF based firewall.  Reference the PF tutorial.
For a firewall, an embedded computer with a few network interfaces has been
installed with OpenBSD.  The Soekris units work very well with OpenBSD.  Since
the firewall is a dedicated unit, and will only be running pf on top of the base
kernel, most of base components don't need to be installed.  The following can
be left out during the firewall installation:
\begin{itemize}
\item
game??.tgz
\item
man??.tgz
\item
xbase??.tgz
\item
xetc??.tgz
\item
xshare??.tgz
\item
xfont??.tgz
\item
xserv??.tgz
\end{itemize}
The address on the external interface will be assigned via DHCP from the ISP,
but the address on the internal interface will be statically set.  The static
address on the internal interface will help facilitate the initial setup of the
network and make sure that the internet is accessible in the event of failures
elsewhere in the network.

\subsection{PF}
% The most important part of the firewall, the packet filter
Initially the packet filter will be configured to pass the following protocols:
\begin{itemize}
\item
SSH - tcp port 22
\item
DNS - udp port 53
\item
HTTP - tcp port 80
\item
NTP - udp port 123
\item
HTTPS - tcp port 443
\end{itemize}
DHCP traffic should be passed for the firewall itself, as DHCP will be used for
the external interface.

% Add the configuration editing here

% Eventually should use DNS, NTP, and Kerberos from AAA Core, but for now, this
% will be managed independently.

\section{WiFi}
% Setup a fairly secure WiFi network that is isolated from the home automation
% systems network.

\section{PBX}
% Setup a PBX (asterisk? or freepbx?)

\subsection{??}
% Setup the PBX itself

\subsection{SIP Handsets}
% Setup the old Aircell SIP handsets (one wired and one wireless minimum)



% Chapter - User Interface
\chapter{User Interface}
% Discuss the importance of user interfaces and basic security issues with them.

\section{Web GUI Interface}
% Setup a web GUI to start being able to monitor and control the system (from
% something easier than SSH)

\section{Text Command}
% Setup the good old WYSE 50 terminal, because it's just so damn cool

\section{Status Display}
% Build some status and text displays for deployment around the house

\section{Status Announcement}
% Setup voice status announcement similar to Star Trek

\section{Voice Command}
% Setup voice control.  This will probably be one of the last things to tackle
% will probably be a forever ongoing task.



% Chapter - Environmental Monitoring and Control
\chapter{Environmental Monitoring and Control}
% Discuss some on the challenges and security issue of environmental fun.
% Include plans for failsafes in each of these sections!

\section{Scheduling}
% Setup a system for making things happen that's more comprehensive than cron.
% This will focus on environment items, system maintenance will still be
% handled by cron on the systems that the maintenance needs to be run.

\section{Lighting}
% Power (AC vs DC)
% Efficiency (Incandescent, fluorescent, CFL, LED) and (AC transmission and
%    conversion to DC vs DC transmission and voltage stepping)
% Control (Color and Brightness)

\section{HVAC}
% Controlling the units (Furnace and AC)
% Temperature monitoring
% Setting the control points and varying on a schedule.

\section{Occupancy Monitoring}
% Is someone home? And how does the system act if yes/no?
% Who's home?



% Chapter - Entertainment
\chapter{Entertainment}
% These are kinda gimmicky, but will probably be fun to try.  These will
% probably require the occupancy monitoring stuff.

\section{Whole House Stereo}
% Music that follows people.

\section{Whole House Video}
% Video on monitors that follows people.



\appendix
% Appendix Chapter - Hardware Support Notes
% This chapter should be written to be publicly distributable.
\chapter{Hardware Specific Notes}
Many pieces of hardware require knowledge of certain quirks and oddities to be able to deploy said hardware into a
particular implementation.  This chapter is a catch all for oddities about various pieces of hardware used through out
this home automation project.

\section{Inovonics RDL-8500}
% Inovonics RDL-8500 used originally for submetering (OpenBSD does not like this machine's serial ports)

% PUT A PICTURE HERE

\subsection{OpenBSD and serial ports}
The single board computer inside the RDL-8500 contains a PCI bus based serial port expansion device (made on to the
board instead of as a PCI card).  This device does not play well with the standard serial port driver during boot.  The
issues observed thus far manifest themselves as a system hang at the following message:
\begin{verbatim}
Setting tty flags=
\end{verbatim}
The most elegant solution that has been proposed is to remove the lines from the \verb=/etc/ttys= file pertaining to the
serial ports of the serial port expansion device.  This prevents the command \verb=ttyflags -a= from acting on poorly
behaved device during boot.  It is easiest to comment out these lines by booting to the ramdisk kernel and using sed to
comment out the lines.  An example follows:
\begin{verbatim}
boot> bsd.rd
...
# fsck /dev/wd0a
...
# mount /dev/wd0a /mnt
# sed 's/tty02/#tty02' /mnt/etc/ttys > /mnt/etc/ttys.tmp
# sed 's/tty03/#tty03' /mnt/etc/ttys.tmp > /mnt/etc/ttys.tmp2
# sed 's/tty04/#tty04' /mnt/etc/ttys.tmp2 > /mnt/etc/ttys.tmp
# sed 's/tty05/#tty05' /mnt/etc/ttys.tmp > /mnt/etc/ttys.tmp2
# sed 's/tty06/#tty06' /mnt/etc/ttys.tmp2 > /mnt/etc/ttys.tmp
# sed 's/tty07/#tty07' /mnt/etc/ttys.tmp > /mnt/etc/ttys.tmp2
# less /mnt/etc/ttys.tmp2
...
# mv /mnt/etc/ttys.tmp2 /mnt/etc/ttys
# umount /mnt
# reboot
...
\end{verbatim}
% The two temporary files is required because the file will be blank is sed output is redirected back to the input file.
% NOTE: When moving from OpenBSD 5.3 to OpenBSD 5.4, the ports moved and required more to be included in the list.
Once the system reboots, the system should no longer hang and should progress to a login prompt.  Keep in mind that the
system will no longer accept dial-in sessions on the serial ports connected to the pci expansion device.  Any dial-in
connection or terminal sessions will have to use one of the two serial ports built into the single board computer (i.e.
\verb=/dev/tty00= and/or \verb=/dev/tty01=).

\section{Flash Memory}
% Discuss flash memory helpers that aren't specific to peticular devices

\subsection{Temporary Filesystem in RAM}
% Discuss moving /tmp to a ramdisk

\subsubsection{OpenBSD}

\subsubsection{Debian Linux}

% Appendix Chapter - My Configurations
% When compiling for public distribution, don't include this section or swap it
% with a version that uses example.com
%\include{chapter_mc/mc.inc}

\backmatter
\chapter{Last Notes}
Don't copy and paste.  If you are implementing any of this, then it is most
likely your own home or place of work, and you will want to make sure you
understand the systems that you are working on.  Make sure you understand the
systems and procedures and make sure you only implement what you know you can
securely implement.  If you do it wrong, it's YOUR happy ass!

\end{document}
