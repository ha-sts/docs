\chapter{Hardware Specific Notes}
Many pieces of hardware require knowledge of certain quirks and oddities to be able to deploy said hardware into a
particular implementation.  This chapter is a catch all for oddities about various pieces of hardware used through out
this home automation project.

\section{Inovonics RDL-8500}
% Inovonics RDL-8500 used originally for submetering (OpenBSD does not like this machine's serial ports)

% PUT A PICTURE HERE

\subsection{OpenBSD and serial ports}
The single board computer inside the RDL-8500 contains a PCI bus based serial port expansion device (made on to the
board instead of as a PCI card).  This device does not play well with the standard serial port driver during boot.  The
issues observed thus far manifest themselves as a system hang at the following message:
\begin{verbatim}
Setting tty flags=
\end{verbatim}
The most elegant solution that has been proposed is to remove the lines from the \verb=/etc/ttys= file pertaining to the
serial ports of the serial port expansion device.  This prevents the command \verb=ttyflags -a= from acting on poorly
behaved device during boot.  It is easiest to comment out these lines by booting to the ramdisk kernel and using sed to
comment out the lines.  An example follows:
\begin{verbatim}
boot> bsd.rd
...
# fsck /dev/wd0a
...
# mount /dev/wd0a /mnt
# sed 's/tty02/#tty02' /mnt/etc/ttys > /mnt/etc/ttys.tmp
# sed 's/tty03/#tty03' /mnt/etc/ttys.tmp > /mnt/etc/ttys.tmp2
# sed 's/tty04/#tty04' /mnt/etc/ttys.tmp2 > /mnt/etc/ttys.tmp
# sed 's/tty05/#tty05' /mnt/etc/ttys.tmp > /mnt/etc/ttys.tmp2
# sed 's/tty06/#tty06' /mnt/etc/ttys.tmp2 > /mnt/etc/ttys.tmp
# sed 's/tty07/#tty07' /mnt/etc/ttys.tmp > /mnt/etc/ttys.tmp2
# less /mnt/etc/ttys.tmp2
...
# mv /mnt/etc/ttys.tmp2 /mnt/etc/ttys
# umount /mnt
# reboot
...
\end{verbatim}
% The two temporary files is required because the file will be blank is sed output is redirected back to the input file.
% NOTE: When moving from OpenBSD 5.3 to OpenBSD 5.4, the ports moved and required more to be included in the list.
Once the system reboots, the system should no longer hang and should progress to a login prompt.  Keep in mind that the
system will no longer accept dial-in sessions on the serial ports connected to the pci expansion device.  Any dial-in
connection or terminal sessions will have to use one of the two serial ports built into the single board computer (i.e.
\verb=/dev/tty00= and/or \verb=/dev/tty01=).

\section{Flash Memory}
% Discuss flash memory helpers that aren't specific to peticular devices

\subsection{Temporary Filesystem in RAM}
% Discuss moving /tmp to a ramdisk

\subsubsection{OpenBSD}

\subsubsection{Debian Linux}