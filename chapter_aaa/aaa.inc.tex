\chapter{Authentication, Authorization, and Accounting}
% Explain AAA and why we need it

\section{Dependencies}

\subsection{OpenNTPD}
% Talk about a basic OpenNTPD setup
One of the key components to any \gls{aaa} system is time.  Determining when a
users' access is valid is integral to maintaining a secure system.  This helps
to prevent replay attacks and prevent access using past (or future) expired
credentials.

\gls{openbsd} prefers the \gls{openntpd} implementation of the \gls{ntp} service
because it is a free (as in freedom) and simple implementation.  Initially,
\gls{openntpd} will be setup on the \gls{aaa} server in order to keep the
\gls{aaa} services better in sync with the clock.  In the future, more advanced
time servers will be setup.  At that point, the \gls{openntpd} server on the
\gls{aaa} server will be setup to sync from those time servers.

In order to activate a basic setup of \gls{openntpd} on \gls{openbsd}, the
following line should be entered into the \verb=/etc/rc.conf.local= file:
\begin{verbatim}
ntpd_flags=
\end{verbatim}
Upon reboot, the \gls{openntpd} service will start automatically with the system
and keep the clock in sync with the internet time servers.

% Do I add a subsection here or later on for more advanced OpenNTPD topics?
% Things like alternate time sources: GPS, Atomic Radio, Local Atomic (rubidium)

\subsection{Domain Name Service}
% We'll setup a DNS server on the kerberos domain controller.
Another key component to any \gls{aaa} system is name resolution.  Name
resolution allows client machines to find the key server to request credentials
for access.  Without being able to find the key server, the key server is rather
useless.  A number of people will say that it's better to hard code the IP
address of the key server into the clients, but DNS keeps us from having to
manually update that configuration.  Hard set IP addresses are not more secure
than DNS, as both can be spoofed and modified on the network.

% At some future point, move to another server with CARP and management scripts.

\subsection{Dynamic Host Configuration Protocol}
% Basic DHCP setup.  Should also make it handle IPv6 and putting the names into
% DNS.

% At some future point, move to another server with CARP and management scripts
% (probably the same servers as DNS)

\section{Kerberos 5}
% Mention that Kerberos serves the Authentication role
% Discuss a basic kerberos setup

% Do I add subsections here for things like redundancy (CARP) here?

\section{Network Information Service}
% Mention that NIS/YP server the Authorization role
% Lets use NIS/YP over LDAP

% Do I add subsections here for things like redundancy (CARP) here?

\section{Accounting - ??}
% Is syslog sufficient?

% Do I add subsections here for things like redundancy (CARP) here?

\section{Certificate Authority}
% Does it make sense to put this here?  Most of our connections will be secured
% via SSL, so we should have a CA

