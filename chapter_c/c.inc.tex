\chapter{Communication}

\section{Firewall}
% Setup an OpenBSD PF based firewall.  Reference the PF tutorial.
For a firewall, an embedded computer with a few network interfaces has been
installed with OpenBSD.  The Soekris units work very well with OpenBSD.  Since
the firewall is a dedicated unit, and will only be running pf on top of the base
kernel, most of base components don't need to be installed.  The following can
be left out during the firewall installation:
\begin{itemize}
\item
game??.tgz
\item
man??.tgz
\item
xbase??.tgz
\item
xetc??.tgz
\item
xshare??.tgz
\item
xfont??.tgz
\item
xserv??.tgz
\end{itemize}
The address on the external interface will be assigned via DHCP from the ISP,
but the address on the internal interface will be statically set.  The static
address on the internal interface will help facilitate the initial setup of the
network and make sure that the internet is accessible in the event of failures
elsewhere in the network.

\subsection{PF}
% The most important part of the firewall, the packet filter
Initially the packet filter will be configured to pass the following protocols:
\begin{itemize}
\item
SSH - tcp port 22
\item
DNS - udp port 53
\item
HTTP - tcp port 80
\item
NTP - udp port 123
\item
HTTPS - tcp port 443
\end{itemize}
DHCP traffic should be passed for the firewall itself, as DHCP will be used for
the external interface.

% Add the configuration editing here

% Eventually should use DNS, NTP, and Kerberos from AAA Core, but for now, this
% will be managed independently.

\section{WiFi}
% Setup a fairly secure WiFi network that is isolated from the home automation
% systems network.

\section{PBX}
% Setup a PBX (asterisk? or freepbx?)

\subsection{??}
% Setup the PBX itself

\subsection{SIP Handsets}
% Setup the old Aircell SIP handsets (one wired and one wireless minimum)

